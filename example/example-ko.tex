\documentclass[ko]{snu-cse-ms-thesis}

% Add your packages here, e.g.,
% \usepackage{tikz}
\usepackage{siunitx}

% For lorem ipsum; remove these lines when writing your thesis
\usepackage{lipsum}
\usepackage{jiwonlipsum}

% hyperref *must* be the last package to be loaded!
\usepackage[pdfusetitle]{hyperref}

\addbibresource{bib.bib}

\title{\textsc{React-tRace:} 리액트 훅을 이해하기 위한 의미구조}
\titlealt{\textsc{React-tRace:} A Semantics for Understanding React Hooks}
\author{이재호}
\advisor{이광근}
\committee{위원장}{이광근}{교수님}
\date{2025년 11월}
\graduationdate{2026년 2월}
\approvaldate{2025년 12월}

\koreankeywords{실행 의미구조, 함수형 언어, 그래픽 사용자 인터페이스 언어, 엄밀한 언어 정의, 시각화 시스템 및 도구, 리액트, 훅, 그리기 의미구조, 함수형 반응형 프로그래밍}
\englishkeywords{operational semantics, functional languages, graphical user interface languages, formal language definitions, visualization systems and tools, react, hooks, render semantics, functional reactive programming}


\begin{document}
\maketitle

\pagenumbering{roman}
\begin{abstract}
오늘날 웹 앞단을 만드는 데 가장 널리 쓰이는 개발 틀인 리액트는, 사용자 화면을 선언적이고 조립식으로 만들 수 있게 해 준다.
훅은 함수형 화면 부품에서 부수 효과를 관리하는 API 모음이다.
그러나 훅의 동작 방식은 개발자에게 종종 이해하기 어렵게 느껴져, 화면을 그릴 때 버그로 이어지곤 한다.
우리는 리액트 훅의 핵심 의미구조를 엄밀하게 담아낸 \textsc{React‑tRace}를 만들어서 훅의 동작 방식을 명확히 한다.
우리 모델이 리액트의 실제 동작을 잘 담아낸다는 것을 보이기 위해, 이론적으로는 훅의 중요한 특성을 만족함을 보이고, 실험적으로는 \textsc{React‑tRace} 실행기를 테스트 모음과 비교한다.
더 나아가, 엄밀하게 정의한 의미구조를 기반으로 실용적인 시각화 도구를 구현하여 개발자가 훅의 동작 방식을 더 잘 이해하도록 돕는다.
\end{abstract}

\tableofcontents
\listoftables
\listoffigures

\chapter{서론}\label{chap:introduction}
\pagenumbering{arabic}
본 템플릿의 구성은 다음과 같다.
\ref{chap:body}장 본론의 \ref{sec:picture}절에서 그림의 예시를 보여준다.
\ref{sec:table}절에서 표의 예시를 보여준다.
\ref{chap:conclusion}장에서는 본 템플릿을 요약한다.

\section{절 예시}\label{sec:section}
\jiwon[2-3]


\chapter{본론}\label{chap:body}
정보 엔트로피는 각 메시지에 포함된 정보의 기댓값으로 식~\eqref{eq:entropy}\와 같다~\cite{6773024}.
\begin{equation}\label{eq:entropy}
  H(X) = -\sum_{i=1}^n {\mathrm{P}(x_i) \log_b \mathrm{P}(x_i)}
\end{equation}

\jiwon[4-6]


\section{그림}\label{sec:picture}
그림 예시는 그림~\ref{fig:example}\와 같다. 그림~\ref{fig:snu}\은 서울대학교 로고이고 그림~\ref{fig:eng}\는 서울대학교 공과대학 로고이다.

\begin{figure}[htp]
  \centering
  \begin{subfigure}[b]{0.5\textwidth}
    \centering
    \includegraphics[width=0.5\textwidth]{logo1.pdf}
    \bicaption{서울대학교 로고}{The logo of Seoul National University}\label{fig:snu}
  \end{subfigure}%
  \begin{subfigure}[b]{0.5\textwidth}
    \centering
    \includegraphics[width=0.9\textwidth]{logo2.pdf}
    \bicaption{공과대학 로고}{The logo of College of Engineering}\label{fig:eng}
  \end{subfigure}
  \bicaption[그림 예시 (목차 항목)]{그림 예시.}{An example of a figure.}\label{fig:example}
\end{figure}

\jiwon[7-8]


\section{표}\label{sec:table}
표 예시는 표~\ref{tab:example}\과 같다.\footnote{\jiwon[12]}

\begin{table}[htp]
  \centering
  \bicaption[표 예시 (목차 항목)]{표 예시.}{An example of a table.}\label{tab:example}
  \begin{tblr}{cc}
    \toprule
    상수 & 값 \\\midrule
    $c$ & \SI{299792458}{\meter\per\second} \\
    $h$ & \SI{6.62607015e-34}{\joule\per\hertz} \\\bottomrule
  \end{tblr}
\end{table}

\jiwon[9-10]


\chapter{결론}\label{chap:conclusion}
\jiwon[11]

\printbibliography

\begin{abstract}[en]
React has become the most widely used web front-end framework, enabling the creation of user interfaces in a declarative and compositional manner.
Hooks are a set of APIs that manage side effects in function components in React.
However, their semantics are often seen as opaque to developers, leading to UI bugs.
We introduce \textsc{React-tRace}, a formalization of the semantics of the essence of React Hooks, providing a semantics that clarifies their behavior.
We demonstrate that our model captures the behavior of React, by theoretically showing that it embodies essential properties of Hooks and empirically comparing our \textsc{React-tRace}-definitional interpreter against a test suite.
Furthermore, we showcase a practical visualization tool based on the formalization to demonstrate how developers can better understand the semantics of Hooks.
\end{abstract}
\end{document}
